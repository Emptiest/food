% Compiling using XeLaTeX
\documentclass{article}
	\title{\textbf{Doenjang-Jjigae}}
	\author{Fullest}
	\date{February 9, 2023}
	
	% Functional packages
	\usepackage{todonotes}
		% \listoftodos
		% \todo{TODO as a margin block}
		% \todo[inline]{TODO within the text}
		% \missingfigure{Description of missing figure}
	\usepackage{siunitx}
		\let\DeclareUSUnit\DeclareSIUnit
		\let\US\SI
		\DeclareUSUnit\fahrenheit{F}
		\DeclareUSUnit\cup{cup}
		\DeclareUSUnit\cups{cups}
		\DeclareUSUnit\tbsp{tbsp}
		\DeclareUSUnit\tsp{tsp}
		\DeclareUSUnit\florets{florets}

	% Margins and Geometry
	\usepackage{multicol}
	\usepackage[margin=0.5in]{geometry}
	\addtolength{\topmargin}{-.5in}
	\usepackage{enumitem}
	\setlist{nosep}
	\usepackage[all]{nowidow}
	\hyphenpenalty=5000
	
	% Remove paragraph indentation
	\setlength{\parindent}{0pt}
	
	% Bibliography Spacing
	\newlength{\bibitemsep}\setlength{\bibitemsep}{.2\baselineskip plus .05\baselineskip minus .05\baselineskip}
	\newlength{\bibparskip}\setlength{\bibparskip}{0pt}
	\let\oldthebibliography\thebibliography
	\renewcommand\thebibliography[1]{%
  		\oldthebibliography{#1}%
  		\setlength{\parskip}{\bibitemsep}%
  		\setlength{\itemsep}{\bibparskip}%
	}
	
	% Helps avoid orphaned words. To use, do {\microtypecontext{expansion=sloppy}blahblahblah}
	\usepackage{microtype}
	\SetExpansion
	[ context = sloppy,
	  stretch = 30,
	  shrink = 60,
	  step = 5 ]
	{ encoding = {OT1,T1,TS1} }
	{ }
	
	% Dark Theme
	\usepackage{xcolor}
	\pagecolor[rgb]{0,0,0}
	\color[rgb]{0.75,0.75,0.75}
	\usepackage[colorlinks = true,
				linkcolor = cyan,
				urlcolor = cyan,
				citecolor = cyan,
				anchorcolor = cyan]{hyperref}	

	\bibliographystyle{unsrt}
	
\begin{document}

\maketitle
\thispagestyle{empty}

\begin{multicols}{2}

% \listoftodos

Updated \today
\section{Intro}
Serves: 4 people\\
Cooking time: ~30 minutes\\

I've made donejang-jjigae\cite{MAANGCHI} many times before, but I'm combining it with instant pot\cite{FISHSTOCK} usage this time. This twist on the recipe cuts down roughly 45m-1h of active cooking time (constant stirring, etc.) to ~30 minutes of mostly passive cooking, not including prep.\\

The stew, if re-heated to a simmer daily, should last a few days but probably not longer than a week. I've never let it sit around for more than 4-5 days, so not tested beyond that shelf life. If the stew starts looking dry after preserving it this way for a few days, add a cup of water.

\section{Ingredients List}
\begin{itemize}
	\item \US{5}{cups} water
	\item 10 anchovies
	\item 2 potatoes
	\item 2 zucchinis
	\item 2 onions
	\item 8 garlic cloves, minced
	\item 1/2 cup of fermented soybean paste (doenjang)
	\item 1 package of medium-firm tofu
\end{itemize}

\section{Tools used}
\begin{itemize}
	\item Knife
	\item Cutting board
	\item Disposable teabag
	\item Instant Pot
	\item Wooden Spoon
	\item Ladle
\end{itemize}
\vfill\null
\columnbreak

\section{Cooking Method}
\begin{itemize}
	\item \verb|Add| Water to Instant Pot
	\item \verb|Put| Anchovies into a looseleaf teabag - this will be used to fish them out later without creating a mess
	\item \verb|Chop| Potatoes, Onions, Garlic, and Zucchini
	\item \verb|Add| Anchovy bag, Potatoes, Onions, Garlic, and Zucchini to Instant Pot
	\item \verb|Cook| on high pressure for 5 minutes + 5 minute natural release
	\item \verb|Remove| Anchovy bag from the broth and \verb|Discard|
	\item \verb|Saute on High| on the Instant Pot and wait for it to simmer
	\item \verb|Chop| Tofu into bite-sized chunks, approx \SI{1.5}{cm} cubes
	\item \verb|Add| Tofu to the instant pot
	\item \verb|Boil| for 3 minutes
\end{itemize}

\section{Serving Method}
Serve with rice, according to your preference. My preference is to keep the rice in a separate bowl, but you can also serve with a bowl of this stew and adding rice on top.

\bibliography{ref}

\end{multicols}
\end{document}